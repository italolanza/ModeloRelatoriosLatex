\documentclass[a4paper, 14pt]{article}
\usepackage[utf8]{inputenc}
\usepackage[brazilian]{babel}
\usepackage[top = 3cm, right = 3cm, left = 2cm, bottom = 2cm]{geometry}
\usepackage{natbib}
\usepackage{url}
\usepackage{amsmath}
\usepackage{graphicx}
\usepackage{parskip}
\usepackage{fancyhdr}
\usepackage{indentfirst}
\usepackage{amssymb}
\usepackage{float}
\usepackage{multicol}
\usepackage{pdfpages}
\usepackage{hhline}
\usepackage{blindtext}

\graphicspath{{figuras/}} %define onde procurar as imagens


\title{Relatório - Experimento X}					% Title
\author{Autor 1\\ Autor 2\\Autor 3\\Autor 4}		% Author
\date{\today}										% Date

\makeatletter
\let\thetitle\@title
\let\theauthor\@author
\let\thedate\@date
\makeatother

\pagestyle{fancy}
\fancyhf{}
\lhead{\thetitle}
\cfoot{\thepage}



\begin{document}

%%%%%%%%%%%%%%%%%%%% Início da Capa %%%%%%%%%%%%%%%%%%%%%%%

\begin{titlepage}
	\centering
    \vspace*{0 cm}
    \includegraphics[scale = 0.3]{logo_ufabc.png}\\[1.0 cm]	% University Logo
    \textsc{\LARGE Universidade Federal do ABC}\\[2.0 cm]	% University Name
	\textsc{\Large Número da Disciplina}\\[0.5 cm]				% Course Code
	\textsc{\large Nome da disciplína}\\[0.5 cm]				% Course Name
	\rule{\linewidth}{0.2 mm} \\[0.4 cm]
	{ \huge \bfseries \thetitle}\\
	\rule{\linewidth}{0.2 mm} \\[1.5 cm]
	
	\begin{minipage}{0.4\textwidth}
		\begin{flushleft} \large
			\emph{Alunos: }\\
			\theauthor
			\end{flushleft}
			\end{minipage}~
			\begin{minipage}{0.4\textwidth}
			\begin{flushright} \large
			\emph{R.A.:} \\
			RA1 \\ RA2\\ RA3\\ RA4									% Your Student Number
		\end{flushright}
	\end{minipage}\\[2 cm]
	
	{\large \thedate}\\[2 cm]
 
	\vfill	
\end{titlepage}




%%%%%%%%%%%%%%%%%%%% Início do Resumo %%%%%%%%%%%%%%%%%%%%%%%

\thispagestyle{empty}
\begin{abstract}

\blindtext %texto enche linguiça

\end{abstract}
\pagebreak


%%%%%%%%%%%%%%%%%%%% Início do Sumário %%%%%%%%%%%%%%%%%%%%%%%

\tableofcontents
\thispagestyle{empty}
\pagebreak



%%%%%%%%%%%%%%%%%%%% Início: Introdução %%%%%%%%%%%%%%%%%%%%%%%
\section{Introdução}
\blindtext %texto enche linguiça


\pagebreak
%%%%%%%%%%%%%%%%%%%% Início: Objetivo %%%%%%%%%%%%%%%%%%%%%%%%%%
\section{Objetivos}
\blindtext %texto enche linguiça

\pagebreak
%%%%%%%%%%%%%%%% Início: Materiais utilizados %%%%%%%%%%%%%%%%%%%%%%%
\section{Materiais utilizados}
\blindtext %texto enche linguiça

\pagebreak

%%%%%%%%%%%%%%%%%%%% Início: Metodologia %%%%%%%%%%%%%%%%%%%%%%%%%%
\section{Metodologia}
\blindtext %texto enche linguiça

\pagebreak

%%%%%%%%%%%%%%%%%%%% Início: Resultados %%%%%%%%%%%%%%%%%%%%%%%%%%
\section{Resultados e discussões}
\blindtext %texto enche linguiça

\pagebreak

%%%%%%%%%%%%%%%%%%%% Início: Conclusão %%%%%%%%%%%%%%%%%%%%%%%%%%
\section{Conclusão}
\blindtext %texto enche linguiça

\pagebreak


\newpage
%\nocite{*} %para citar todas as referências no arquivo de referências
\bibliographystyle{plain}
\bibliography{referencias}

\pagebreak

%\appendix %anexos (opicional)



\end{document}
